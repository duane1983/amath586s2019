\documentclass[10pt]{article}

\usepackage{graphicx}
\usepackage{amsmath,amsfonts,amssymb}

\usepackage{hyperref}  % for urls and hyperlinks


\setlength{\textwidth}{6.2in}
\setlength{\oddsidemargin}{0.3in}
\setlength{\evensidemargin}{0in}
\setlength{\textheight}{8.9in}
\setlength{\voffset}{-1in}
\setlength{\headsep}{26pt}
\setlength{\parindent}{0pt}
\setlength{\parskip}{5pt}




% a few handy macros

\newcommand\matlab{{\sc matlab}}
\newcommand{\goto}{\rightarrow}
\newcommand{\bigo}{{\mathcal O}}
\newcommand{\half}{\frac{1}{2}}
%\newcommand\implies{\quad\Longrightarrow\quad}
\newcommand\reals{{{\rm l} \kern -.15em {\rm R} }}
\newcommand\complex{{\raisebox{.043ex}{\rule{0.07em}{1.56ex}} \hskip -.35em {\rm C}}}


% macros for matrices/vectors:

% matrix environment for vectors or matrices where elements are centered
\newenvironment{mat}{\left[\begin{array}{ccccccccccccccc}}{\end{array}\right]}
\newcommand\bcm{\begin{mat}}
\newcommand\ecm{\end{mat}}

% matrix environment for vectors or matrices where elements are right justifvied
\newenvironment{rmat}{\left[\begin{array}{rrrrrrrrrrrrr}}{\end{array}\right]}
\newcommand\brm{\begin{rmat}}
\newcommand\erm{\end{rmat}}

% for left brace and a set of choices
\newenvironment{choices}{\left\{ \begin{array}{ll}}{\end{array}\right.}
\newcommand\when{&\text{if~}}
\newcommand\otherwise{&\text{otherwise}}
% sample usage:
%  \delta_{ij} = \begin{choices} 1 \when i=j, \\ 0 \otherwise \end{choices}


% for labeling and referencing equations:
\newcommand{\eql}{\begin{equation}\label}
\newcommand{\eqn}[1]{(\ref{#1})}
% can then do
%  \eql{eqnlabel}
%  ...
%  \end{equation}
% and refer to it as equation \eqn{eqnlabel}.  


% some useful macros for finite difference methods:
\newcommand\unp{U^{n+1}}
\newcommand\unm{U^{n-1}}

% for chemical reactions:
\newcommand{\react}[1]{\stackrel{K_{#1}}{\rightarrow}}
\newcommand{\reactb}[2]{\stackrel{K_{#1}}{~\stackrel{\rightleftharpoons}
   {\scriptstyle K_{#2}}}~}

% Parts:

% set enumerate to give parts a, b, c, ...  rather than numbers 1, 2, 3...
\renewcommand{\theenumi}{\alph{enumi}}
\renewcommand{\labelenumi}{(\theenumi)}

% set second level enumerate to give parts i, ii, iii, iv, etc.
\renewcommand{\theenumii}{\roman{enumii}}
\renewcommand{\labelenumii}{(\theenumii)}

  % input some useful macros

\begin{document}

% header:
\hfill\vbox{\hbox{AMath 586 / ATM 581}
\hbox{Homework \#1}\hbox{Due Thursday, April 11, 2019}}

\vskip 5pt

Homework is due to Canvas by 11:00pm PDT on the due date.

To submit, see
\url{https://canvas.uw.edu/courses/1271892/assignments/4773426}


%--------------------------------------------------------------------------
\vskip 1cm
\hrule

If you haven't done so already, clone the class repository and read about how 
to use it on the class webpage
\url{http://staff.washington.edu/rjl/classes/am586s2016/class_repos.html}.

Also figure out how to use Jupyter notebooks from the {\tt notebooks} directory
of the repository.  See 
\url{http://staff.washington.edu/rjl/classes/am586s2016/code.html}.



%--------------------------------------------------------------------------
\vskip 1cm
\hrule
{\bf Problem 1.}

Check that the solution $u(t)$ given by (5.8) in the textbook
satisfies the ODE (5.6) and
initial condition.  Hint: To differentiate the matrix exponential you can
differentiate the Taylor series (D.31) (in Appendix D) term by term.

% uncomment the next two lines if you want to insert solution...
%\vskip 1cm
%{\bf Solution:}

% insert your solution here!

%--------------------------------------------------------------------------
\vskip 1cm
\hrule
{\bf Problem 2.}


Consider the IVP
\begin{equation*}
\begin{split}
u_1' &= 2u_1,\\
u_2' &= 3u_1 - u_2,
\end{split}
\end{equation*}
with initial conditions specified at time $t=0$.  Solve this problem in two
different ways:

\begin{enumerate}
\item Solve the first equation, which only involves $u_1$, and then insert
this function into the second equation to obtain a nonhomogeneous linear
equation for $u_2$.  Solve this using (5.8).  
Check that your solution satisfies the initial conditions and the ODE.

\item Write the system as $u' = Au$ and compute the matrix exponential using
(D.30) to obtain the solution.
\end{enumerate}


% uncomment the next two lines if you want to insert solution...
%\vskip 1cm
%{\bf Solution:}

% insert your solution here!


%--------------------------------------------------------------------------
\vskip 1cm
\hrule
{\bf Problem 3.}


Consider the IVP
\begin{equation*}
\begin{split}
u_1' &= 2u_1,\\
u_2' &= 3u_1 + 2u_2,
\end{split}
\end{equation*}
with initial conditions specified at time $t=0$.  Solve this problem in two
different ways:

\begin{enumerate}
\item Solve the first equation, which only involves $u_1$, and then insert
this function into the second equation to obtain a nonhomogeneous linear
equation for $u_2$.  Solve this using (5.8).
Check that your solution satisfies the initial conditions and the ODE.

\item Write the system as $u' = Au$ and compute the matrix exponential using
(D.35) to obtain the solution.  (See Appendix C.3 for a discussion of the
Jordan Canonical form in the defective case.)
\end{enumerate}


% uncomment the next two lines if you want to insert solution...
%\vskip 1cm
%{\bf Solution:}

% insert your solution here!

%--------------------------------------------------------------------------
\vskip 1cm
\hrule
{\bf Problem 4.}

Section 2.16 of the textbook describes the ODE modeling a simple pendulum,
given in (2.74):
\begin{equation}\label{pendulum}
\theta''(t) = -(g/L)\sin(\theta(t)).
\end{equation}
with initial data on the angle $\theta(0) = \theta_0$ and the angular velocity $\theta'(0) = v_0$.  


Recall that $\theta$ is the angle from vertical, with $\theta = 0$ (or 
more generally $\theta = 2\pi N$ for any integer $N$) corresponding straight
down.  With velocity
$v=0$ these points are stable equilibria.
The values $\theta = (2N+1)\pi$ correspond to straight up and are 
unstable equilibria is $v=0$.  (Stable in the sense of ODE theory --- if the
data is perturbed by a small amount then the solution stays close to the
original solution.)

For $\theta(0)$ and $\theta'(0)$ sufficiently small this can be approximated
by the linearized equation
\[
\theta''(t) = -(g/L)\theta(t).
\]
Rewrite this as a first order system of two equations and solve using the
matrix exponential approach.  (You probably already know what the solution
looks like in terms of sine and cosine, and you should see this come out of
complex exponentials since the matrix has imaginary eigenvalues in this case.)


% uncomment the next two lines if you want to insert solution...
%\vskip 1cm
%{\bf Solution:}

% insert your solution here!

%--------------------------------------------------------------------------
\vskip 1cm
\hrule
{\bf Problem 5.}

Rewrite the nonlinear pendulum equation (\ref{pendulum}) 
as a first order system of two equations and set up a solver
for this problem in a Jupyter notebook using {\tt scipy.integrate.solve\_ivp}, 
following the example in the notebook {\tt Example\_using\_solve\_ivp} from the
class repository.  

Develop the notebook to illustrate the following tests, 
including some commentary about what you observe.  

Use values $g = 9.81$ m/s$^2$ and $L = 1$ meter.

\begin{enumerate}
\item Show that for $\theta(0)$ sufficiently small and $\theta'(0)=0$ the 
solution computed with {\tt solve\_ivp} agrees well with the exact solution of
the linearized equation, as a test that you have implemented things properly.

\item Try $\theta(0) = 3.1,~ \theta'(0) = 0$ and solve up to time $T = 15$
using the {\tt RK45} method with the default error tolerance.  Note that the
pendulum is close to vertical initially.  Comment on whether the solution
seems reasonable.

\item Try other values of $\theta(0)$ that are close to $\pi$ (also 
values slightly larger than $\pi$) and comment on your results.

\item Try setting {\tt rtol} to a smaller value than the default and comment
on whether you can get reasonable solutions with sufficiently small tolerance.
 
\end{enumerate}

% uncomment the next two lines if you want to insert solution...
%\vskip 1cm
%{\bf Solution:}

% insert your solution here!


%--------------------------------------------------------------------------
\newpage
%\vskip 1cm
\hrule
{\bf Problem 6.}

Determine the coefficients $\beta_0,~\beta_1,~\beta_2$ for the third
order, 2-step Adams-Moulton method.  Do this in two different ways:
\begin{enumerate} 
 \item Using the expression for the local truncation error in Section 5.9.1,
 \item Using the relation
 \[
 u(t_{n+2}) = u(t_{n+1}) + \int_{t_{n+1}}^{t_{n+2}}\,f(u(s))\,ds.
 \]
 Interpolate  a quadratic polynomial $p(t)$ through the three values
 $f(U^n),~f(U^{n+1})$ and $f(U^{n+2})$ and then integrate this polynomial
 exactly to obtain the formula.  The coefficients of the polynomial will
 depend on the three values $f(U^{n+j})$.   It's easiest to use the
 ``Newton form'' of the interpolating polynomial and consider the three
times $t_n=-k$, $t_{n+1}=0$, and $t_{n+2}=k$ so that $p(t)$ has the form
\[
p(t) = A + B(t+k) + C(t+k)t
\]
where $A,~B$, and $C$ are the appropriate divided differences based on the
data.  Then integrate from $0$ to $k$.   (The method has the same
coefficients at any time, so this is valid.)
\end{enumerate}


% uncomment the next two lines if you want to insert solution...
%\vskip 1cm
%{\bf Solution:}

% insert your solution here!

%--------------------------------------------------------------------------

\end{document}
