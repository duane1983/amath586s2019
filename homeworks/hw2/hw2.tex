\documentclass[10pt]{article}

\usepackage{graphicx}
\usepackage{amsmath,amsfonts,amssymb}

\usepackage{hyperref}  % for urls and hyperlinks


\setlength{\textwidth}{6.2in}
\setlength{\oddsidemargin}{0.3in}
\setlength{\evensidemargin}{0in}
\setlength{\textheight}{8.9in}
\setlength{\voffset}{-1in}
\setlength{\headsep}{26pt}
\setlength{\parindent}{0pt}
\setlength{\parskip}{5pt}




% a few handy macros

\newcommand\matlab{{\sc matlab}}
\newcommand{\goto}{\rightarrow}
\newcommand{\bigo}{{\mathcal O}}
\newcommand{\half}{\frac{1}{2}}
%\newcommand\implies{\quad\Longrightarrow\quad}
\newcommand\reals{{{\rm l} \kern -.15em {\rm R} }}
\newcommand\complex{{\raisebox{.043ex}{\rule{0.07em}{1.56ex}} \hskip -.35em {\rm C}}}


% macros for matrices/vectors:

% matrix environment for vectors or matrices where elements are centered
\newenvironment{mat}{\left[\begin{array}{ccccccccccccccc}}{\end{array}\right]}
\newcommand\bcm{\begin{mat}}
\newcommand\ecm{\end{mat}}

% matrix environment for vectors or matrices where elements are right justifvied
\newenvironment{rmat}{\left[\begin{array}{rrrrrrrrrrrrr}}{\end{array}\right]}
\newcommand\brm{\begin{rmat}}
\newcommand\erm{\end{rmat}}

% for left brace and a set of choices
\newenvironment{choices}{\left\{ \begin{array}{ll}}{\end{array}\right.}
\newcommand\when{&\text{if~}}
\newcommand\otherwise{&\text{otherwise}}
% sample usage:
%  \delta_{ij} = \begin{choices} 1 \when i=j, \\ 0 \otherwise \end{choices}


% for labeling and referencing equations:
\newcommand{\eql}{\begin{equation}\label}
\newcommand{\eqn}[1]{(\ref{#1})}
% can then do
%  \eql{eqnlabel}
%  ...
%  \end{equation}
% and refer to it as equation \eqn{eqnlabel}.  


% some useful macros for finite difference methods:
\newcommand\unp{U^{n+1}}
\newcommand\unm{U^{n-1}}

% for chemical reactions:
\newcommand{\react}[1]{\stackrel{K_{#1}}{\rightarrow}}
\newcommand{\reactb}[2]{\stackrel{K_{#1}}{~\stackrel{\rightleftharpoons}
   {\scriptstyle K_{#2}}}~}

% Parts:

% set enumerate to give parts a, b, c, ...  rather than numbers 1, 2, 3...
\renewcommand{\theenumi}{\alph{enumi}}
\renewcommand{\labelenumi}{(\theenumi)}

% set second level enumerate to give parts i, ii, iii, iv, etc.
\renewcommand{\theenumii}{\roman{enumii}}
\renewcommand{\labelenumii}{(\theenumii)}

  % input some useful macros

\begin{document}

% header:
\hfill\vbox{\hbox{AMath 586 / ATM 581}
\hbox{Homework \#2}\hbox{Due Thursday, April 23, 2019}}

\vskip 5pt

Homework is due to Canvas by 11:00pm PDT on the due date.

To submit, see
\url{https://canvas.uw.edu/courses/1271892/assignments/4790261}


%--------------------------------------------------------------------------

%--------------------------------------------------------------------------
\vskip 1cm
\hrule
{\bf Problem 1}

Which of the following Linear Multistep Methods are convergent?  For 
the ones that are not, are they inconsistent, or not zero-stable, or both?
 \begin{enumerate}
 \item $U^{n+3} = U^{n+1} + 2kf(U^n)$,
 \item $U^{n+2} = \half U^{n+1} + \half U^{n} + 2kf(U^{n+1})$,
 \item $\unp = U^n$, 
 \item $U^{n+4} = U^{n} + \frac 4 3 k(f(U^{n+3})+f(U^{n+2})+f(U^{n+1}))$,
 \item $U^{n+3} = -U^{n+2} + U^{n+1} +U^{n}+2k(f(U^{n+2})+f(U^{n+1}))$.
 \end{enumerate}



% uncomment the next two lines if you want to insert solution...
%\vskip 1cm
%{\bf Solution:}

% insert your solution here!


%--------------------------------------------------------------------------
\vskip 1cm
\hrule
{\bf Problem 3}

\begin{enumerate}
\item Determine the general solution to the linear difference equation
$U^{n+2} = U^{n+1} + U^n$.

\item Determine the solution to this difference equation with the starting
values $U^0=1$, $U^1=1$.  Use this to determine $U^{30}$?  
(Note, these are the {\em Fibonacci numbers}, which of course should all be
integers.)

\item Show that for large $n$ the ratio of successive Fibonacci numbers
$U^n/U^{n-1}$ approaches the ``golden ratio'' $\phi \approx 1.618034$.
\end{enumerate} 

% uncomment the next two lines if you want to insert solution...
%\vskip 1cm
%{\bf Solution:}

% insert your solution here!

%--------------------------------------------------------------------------
\vskip 1cm
\hrule
{\bf Problem 2}
Any $r$-stage Runge-Kutta method applied to $u'=\lambda u$ will give an
expression of the form
\[
U^{n+1} = R(z)U^n
\]
where $z=\lambda k$ and $R(z)$ is a rational function, a ratio of
polynomials in $z$ each having degree at most $r$.  For an explicit method
$R(z)$ will simply be a polynomial of degree $r$ and for an implicit method
it will be a more general rational function.

Since $u(t_{n+1}) = e^z u(t_n)$ for this problem, we expect that a $p$th
order accurate method will give a function $R(z)$ satisfying
\[
\qquad  R(z) = e^z + \bigo(z^{p+1}) \quad\text{as}~z \goto 0.
\]
This indicates that the one-step error is $\bigo(z^{p+1})$ on this problem,
as expected for a $p$th order accurate method.

The explicit
Runge-Kutta method of Example 5.13 is fourth order accurate in general,
so in particular it should exhibit this accuracy when applied to 
$u'(t) = \lambda u(t)$.  Show that in fact when applied to this
problem the method becomes $U^{n+1} = R(z)U^n$ where $R(z)$ is 
a polynomial of degree 4, and that this polynomial agrees with the Taylor
expansion of $e^z$ through $O(z^4)$ terms.

We will see that this function $R(z)$ is also important in the study of 
absolute stability of a one-step method.

% uncomment the next two lines if you want to insert solution...
%\vskip 1cm
%{\bf Solution:}

% insert your solution here!

%--------------------------------------------------------------------------
\vskip 1cm
\hrule
{\bf Problem 2}


Determine the function $R(z)$ described in the previous exercise for the
TR-BDF2 method given in (5.37).  Note that this can be simplified to the
form (8.6), which is given only for the autonomous case but that suffices
for $u'(t) = \lambda u(t)$.  (You might want to convince yourself these are 
the same method).

Confirm that $R(z)$ agrees with $e^z$ to the expected order.

Note that for this implicit method $R(z)$ will be a rational function, so you
will have to expand the denominator in a Taylor series, or use the Neumann
series 
\[
1/(1-\epsilon) = 1 + \epsilon + \epsilon^2 + \epsilon^3 + \cdots.
\]

% uncomment the next two lines if you want to insert solution...
%\vskip 1cm
%{\bf Solution:}

% insert your solution here!



%--------------------------------------------------------------------------
%\vskip 1cm
\hrule
{\bf Problem 4}

The Jupyter notebook {\tt Pendulum\_ForwardEuler.ipynb} gives an implementation of Forward Euler on the nonlinear pendulum problem
\begin{align}
\theta'(t) &= v(t)\\
v'(t) &= -\sin(\theta(t))
\end{align}

Modify this code to implement the Backward Euler method.  Since this is an 
implicit method, you need to solve a nonlinear equation in each step.  
Although it is a system of two equations, you can reduce it to a scalar
nonlinear equation.  In each step we have to solve
\begin{align}
\Theta^{n+1} = \Theta^n + kV^{n+1},\\
V^{n+1} = V^n - \sin(\Theta^{n+1})
\end{align}
and substituting the second equation in the first gives a single equation to 
solve for $\Theta^{n+1}$, of the form $\phi(theta) = 0$ with
$\phi(\theta) = \theta + k^2\sin(\theta) - (\Theta^n - kV^n)$.
This can be done in each time step by defining the function $\phi$ and then
calling the root-finder 
\href{https://docs.scipy.org/doc/scipy/reference/generated/scipy.optimize.fsolve.html}{scipy.optimize.fsolve}.
The value $\Theta^n$ from the previous 
time step is generally a good starting guess for the root. So this gives
a loop like:

\begin{verbatim}
for n in range(0,nsteps):
    phi = lambda theta: theta + dt**2*sin(theta) - (U[0,n]  + dt*U[1,n])
    new_theta = fsolve(phi, U[0,n])
    U[0,n+1] = new_theta
    U[1,n+1] = U[1,n] - dt * sin(new_theta)
\end{verbatim}

Comment on how solutions computed with this method compare to 
those obtained with Forward Euler.  Test with different size time steps
to confirm that you see the expected first order accuracy.
For this you may want to take a much shorter time interval than in the
notebook in order to get the expected asymptotics with a reasonable number
of time steps.  Also note that you cannot necessarily assume the 
reference solution computed with {\tt solve\_ivp} is as accurate as the
tolerance specified, especially over long time intervals!

% uncomment the next two lines if you want to insert solution...
%\vskip 1cm
%{\bf Solution:}

% insert your solution here!


%--------------------------------------------------------------------------
\vskip 1cm
\hrule
{\bf Problem 5}

Repeat the previous problem for the Trapezoidal method.  
Again comment on the behavior of the solution in this case, 
and check the order of accuracy.
% uncomment the next two lines if you want to insert solution...
%\vskip 1cm
%{\bf Solution:}

% insert your solution here!


\end{document}
